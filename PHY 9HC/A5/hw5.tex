% A useful template for typesetting beautiful homework solutions. 
% Also check out Professor Matloff's guide: 
% http://heather.cs.ucdavis.edu/~matloff/LaTeX/HowToCreate.html.
\documentclass{article}

% Packages Used
\usepackage{fancyhdr} % Required for custom headers
\usepackage{lastpage} % Required to determine the last page for the footer
\usepackage{extramarks} % Required for headers and footers
\usepackage{graphicx} % Required to insert images
\usepackage{lipsum} % Used for inserting dummy 'Lorem ipsum' text into the template
\usepackage{comment}  % Used for multi-line commenting
\usepackage{booktabs} % For better looking tables
\usepackage{array}       % for better arrays (eg matrices) in maths
\usepackage{paralist}    % very flexible & customisable lists (eg. enumerate/itemize, etc.)
\usepackage{verbatim}  % adds environment for commenting out blocks of text & for better verbatim
\usepackage{subfig}      % make it possible to include more than one captioned figure/table in a single float
\usepackage{amsthm}   % make proofs look better
\usepackage{amsfonts}
\usepackage{amsmath}
\usepackage{amssymb}
\usepackage{eufrak}      % for fraktur fonts
\usepackage{mathabx}  % for \divides
\usepackage{enumerate} % to get lists enumerated with letters
\usepackage{hyperref}  % to get attractive URLs
\usepackage{bussproofs} % for setting proofs
\usepackage{etoolbox}
\usepackage{enumitem}
\usepackage{algorithm}
\usepackage{algorithmic}
\usepackage{multirow}
\usepackage{longtable}
\usepackage{tikz}
\usepackage{slashed}
\usepackage{braket}
\newcommand{\Intvalx}[2]{\mspace{-1mu}\left.\rule[0pt]{0pt}{\IntHtx}\right|_{\,#1}^{\,#2}}
\newcommand{\tabincell}[2]{\begin{tabular}{@{}#1@{}}#2\end{tabular}}
\providecommand{\bigp}[1]{\bigg( #1 \bigg)}
\providecommand{\Matrix}[1] {\begin{bmatrix} #1 \end{bmatrix}}
% For theorem enviornment
\theoremstyle{definition}
\newtheorem{definition}{Definition}
\newtheorem{theorem}{Theorem}[section]
\newtheorem{corollary}{Corollary}[theorem]
\newtheorem{lemma}{Lemma}

\newtheorem{mathrule}{Rule}
\newtheorem{case}{Case}
\newtheorem{subcase}{Case}[case]

\theoremstyle{plain}
\newtheorem{example}{Example}
\newtheorem{problem}{Problem}[section]

% For improved end of proof formatting
\patchcmd{\endproof}  % <cmd>
  {\endtrivlist}               % <search>
  {\endtrivlist\par\nobreak\vspace*{\dimexpr-\baselineskip-\parskip}\nobreak\noindent\hrulefill}% <replace>
  {}{}                            % <succes><failure>

% Margins
\topmargin=-0.45in
\evensidemargin=0in
\oddsidemargin=0in
\textwidth=6.5in
\textheight=9.0in
\headsep=0.25in 

\linespread{1.1} % Line spacing

% Set up the header and footer
\pagestyle{fancy}
\lhead{PHY 9HC Spring 2016\\ UC Davis - Emilija Pantic} % Top left header
\chead{} % Top center header
\rhead{\firstxmark Anze Wang ID: 912777492\\PHY 9HC A02} % Top right header
\lfoot{\lastxmark} % Bottom left footer
\cfoot{} % Bottom center footer
\rfoot{Page\ \thepage\ of\ \pageref{LastPage}} % Bottom right footer

\setlength\parindent{10pt} % Removes all indentation from paragraphs

% Common boolean operators.
\newcommand*\AND{\wedge}
\newcommand*\OR{\vee}
\newcommand*\NOT{\neg}
\newcommand*\IMPLIES{\implies}
\newcommand*\XOR{\mathbin{\oplus}}


\begin{document}

\begin{center} \bf \LARGE Homework 5\\
\end{center}


\begin {enumerate}[itemindent=30pt,label=\bf Exercise {\arabic*}:]

\item Q7M.6\\
\begin{align*}
	&\Ket{\psi_{0}} = \Matrix{\sqrt{4/5} \\ \sqrt{1/5}} = \sqrt{4/5}\; \Ket{+z} + \sqrt{1/5} \;\Ket{-z}\\
	& \Ket{\psi_{t}} = \sqrt{4/5}e^{-iEt/\hbar}\; \Ket{+z} + \sqrt{1/5} \;\Ket{-z}\\
	&\mid \Braket{+x | \psi_{t}} \mid^2 = \dfrac{2}{5} + \dfrac{1}{5} e^{iEt/\hbar} + \dfrac{1}{5} e^{-iEt/\hbar} + \dfrac{1}{10} = \dfrac{1}{2} + \dfrac{2}{5} cos(Et/\hbar)
\end{align*}
\subitem The probability that we will determine this electron's spin to be in the +x direction at time t is $\dfrac{1}{2} + \dfrac{2}{5} cos(Et/\hbar)$

\item Q7D.4\\
\begin{align*}
	&\Ket{\psi_{0}} = \Matrix{\sqrt{1/2} \\ \sqrt{1/2}} = \sqrt{1/2}\; \Ket{+z} + \sqrt{1/2} \;\Ket{-z}\\
	& \Ket{\psi_{t}} = \sqrt{1/2}e^{-iEt/2\hbar}\; \Ket{+z} + \sqrt{1/2}e^{+iEt/2\hbar} \;\Ket{-z}\\
	&\mid \Braket{+x | \psi_{t}} \mid^2 = \dfrac{1}{4} (e^{iEt/\hbar} + 1 + 1 + e^{-iEt/\hbar}) = \dfrac{1}{2} + \dfrac{1}{2} cos(Et/\hbar)
\end{align*}
\subitem The probability is exactly same as we found in text example.
\item QAM.3\\
\subitem Because $\Ket{\psi(0)} = \Ket{+x}$, therefore  $\ket{\psi(0)} = \sqrt{1/2}\Ket{+z} + \sqrt{1/2}\Ket{-z}$
\subitem In this situation, $\ket{\psi(t)} = \sqrt{1/2} e^{-iE_{+}t/\hbar}\Ket{+z} + \sqrt{1/2} e^{-iE_{-}t/\hbar}\Ket{-z}$
\subitem For $\Braket{x}$ :
\begin{align*}
	\Braket{+x \mid \psi(t)} &= \sqrt{1/2} e^{-iE_{+}t/\hbar}\Braket{+x \mid +z} + \sqrt{1/2} e^{-iE_{-}t/\hbar}\Braket{+x \mid -z}\\
	&= \sqrt{1/2} e^{-iE_{+}t/\hbar}\Matrix{\sqrt{1/2} & \sqrt{1/2}} \Matrix{1 \\ 0}+ \sqrt{1/2} e^{-iE_{-}t/\hbar}\Matrix{\sqrt{1/2} & \sqrt{1/2}} \Matrix{0 \\ 1}\\
	&= \dfrac{1}{2} e^{-iE_{+}t/\hbar}+ \dfrac{1}{2} e^{-iE_{-}t/\hbar}
\end{align*}
\subitem Similarly, $\Braket{-x \mid \psi(t)} = \dfrac{1}{2} e^{-iE_{+}t/\hbar} - \dfrac{1}{2} e^{-iE_{-}t/\hbar}$. Therefore, the probability of these out come is:
\begin{align*}
	|\Braket{-x \mid \psi(t)}|^{2} &= (\dfrac{1}{2} e^{-iE_{+}t/\hbar}- \dfrac{1}{2} e^{-iE_{-}t/\hbar})^{*}(\dfrac{1}{2} e^{-iE_{+}t/\hbar}- \dfrac{1}{2} e^{-iE_{-}t/\hbar})\\ 
	&=(\dfrac{1}{2} e^{iE_{+}t/\hbar} - \dfrac{1}{2} e^{iE_{-}t/\hbar})(\dfrac{1}{2} e^{-iE_{+}t/\hbar}- \dfrac{1}{2} e^{-iE_{-}t/\hbar})\\ 
	&= \dfrac{1}{2} - \dfrac{1}{4}( e^{i(E_{+} - E_{-})t/\hbar} + e^{i(E_{-} - E_{+})t/\hbar})\\
	&= \dfrac{1}{2} - \dfrac{1}{2} cos(\omega t)
\end{align*}
\begin{align*}
	|\Braket{+x \mid \psi(t)}|^{2} &= (\dfrac{1}{2} e^{-iE_{+}t/\hbar} + \dfrac{1}{2} e^{-iE_{-}t/\hbar})^{*}(\dfrac{1}{2} e^{-iE_{+}t/\hbar}+ \dfrac{1}{2} e^{-iE_{-}t/\hbar})\\
	&=\dfrac{1}{2} + \dfrac{1}{2} cos(\omega t)
\end{align*}
\subitem The expectation value for $S_{x}$ is:
\begin{align*}
	\Braket{S_{x}} &= \dfrac{\hbar}{2}(\dfrac{1}{2} + \dfrac{1}{2} cos(\omega t)) - \dfrac{\hbar}{2}(\dfrac{1}{2} - \dfrac{1}{2} cos(\omega t))\\
	&= \dfrac{\hbar}{2} cos(\omega t)
\end{align*}
\subitem For $\Braket{y}$ :
\begin{align*}
	\Braket{+y \mid \psi(t)} &= \sqrt{1/2} e^{-iE_{+}t/\hbar}\Braket{+y \mid +z} + \sqrt{1/2} e^{-iE_{-}t/\hbar}\Braket{+y \mid -z}\\
	&= \sqrt{1/2} e^{-iE_{+}t/\hbar}\Matrix{\sqrt{1/2} & -i\sqrt{1/2}} \Matrix{1 \\ 0}+ \sqrt{1/2} e^{-iE_{-}t/\hbar}\Matrix{\sqrt{1/2} & -i\sqrt{1/2}} \Matrix{0 \\ 1}\\
	&= \dfrac{1}{2} e^{-iE_{+}t/\hbar}-i\dfrac{1}{2} e^{-iE_{-}t/\hbar}
\end{align*}
\subitem Similarly, $\Braket{-y \mid \psi(t)} = \dfrac{1}{2} e^{-iE_{+}t/\hbar} + \dfrac{1}{2} e^{-iE_{-}t/\hbar}$. Therefore, the probability of these out come is:
\begin{align*}
	|\Braket{-y \mid \psi(t)}|^{2} &= (\dfrac{1}{2} e^{-iE_{+}t/\hbar} + i\dfrac{1}{2} e^{-iE_{-}t/\hbar})^{*}(\dfrac{1}{2} e^{-iE_{+}t/\hbar} + i\dfrac{1}{2} e^{-iE_{-}t/\hbar})\\ 
	&=(\dfrac{1}{2} e^{iE_{+}t/\hbar} - i\dfrac{1}{2} e^{iE_{-}t/\hbar})(\dfrac{1}{2} e^{-iE_{+}t/\hbar} +  i\dfrac{1}{2} e^{-iE_{-}t/\hbar})\\ 
	&= \dfrac{1}{2} - \dfrac{i}{4}( e^{i(E_{+} - E_{-})t/\hbar} - e^{i(E_{-} - E_{+})t/\hbar})\\
	&= \dfrac{1}{2} - \dfrac{1}{2} sin(\omega t)
\end{align*}
\begin{align*}
	|\Braket{+y \mid \psi(t)}|^{2} &=(\dfrac{1}{2} e^{-iE_{+}t/\hbar} + \dfrac{1}{2} e^{-iE_{-}t/\hbar})^{*}(\dfrac{1}{2} e^{-iE_{+}t/\hbar} + \dfrac{1}{2} e^{-iE_{-}t/\hbar})\\
	&=\dfrac{1}{2} + \dfrac{1}{2} sin(\omega t)
\end{align*}
\subitem The expectation value for $S_{y}$ is:
\begin{align*}
	\Braket{S_{y}} &= \dfrac{\hbar}{2}(\dfrac{1}{2} + \dfrac{1}{2} sin(\omega t)) - \dfrac{\hbar}{2}(\dfrac{1}{2} - \dfrac{1}{2} sin(\omega t))\\
	&= \dfrac{\hbar}{2} sin(\omega t)
\end{align*}
\subitem for $\Braket{S_{z}}$:
\begin{align*}
	\Braket{+z \mid \psi(t)} &= \sqrt{1/2} e^{-iE_{+}t/\hbar}\Braket{+z \mid +z} + \sqrt{1/2} e^{-iE_{-}t/\hbar}\Braket{+z \mid -z}\\
	&=\sqrt{1/2} e^{-iE_{+}t/\hbar}\\
	\Braket{-z \mid \psi(t)} &= \sqrt{1/2} e^{-iE_{+}t/\hbar}\Braket{-z \mid +z} + \sqrt{1/2} e^{-iE_{-}t/\hbar}\Braket{-z \mid -z}\\
	&=\sqrt{1/2} e^{-iE_{-}t/\hbar}
\end{align*}
\subitem Therefore, we can get the expectation value of $S_{z}$:
\begin{align*}
	\Braket{S_{z}} &= \dfrac{\hbar}{2} |\Braket{+z\mid \psi}|^2 - \dfrac{\hbar}{2} |\Braket{-z \mid \psi}|^2\\
	&=\dfrac{\hbar}{2}
\end{align*}
\subitem According to the calculation, we know that the $\Braket{S_{z}}$ is a constant  and $\Braket{S_x}$ and $\Braket{S_y}$ are functions depend on time. We can easily find that S is rotating about z axis on the xy plane. This motion is what we expect in classic model. So this result make sense. 
\item QAM.6\\
\subitem Because $\Ket{\psi(0)} = \Matrix{4/5 \\ -3/5}$, therefore  $\ket{\psi(0)} = \dfrac{4}{5}\Ket{+z} + -\dfrac{3}{5}\Ket{-z}$
\subitem In this situation, $\ket{\psi(t)} = \dfrac{4}{5} e^{-iE_{+}t/\hbar}\Ket{+z} - \dfrac{3}{5} e^{-iE_{-}t/\hbar}\Ket{-z}$
\subitem For $\Braket{x}$ :
\begin{align*}
	\Braket{+x \mid \psi(t)} &= 4/5 e^{-iE_{+}t/\hbar}\Braket{+x \mid +z} - 3/5 e^{-iE_{-}t/\hbar}\Braket{+x \mid -z}\\
	&= 4/5 e^{-iE_{+}t/\hbar}\Matrix{\sqrt{1/2} & \sqrt{1/2}} \Matrix{1 \\ 0} - 3/5 e^{-iE_{-}t/\hbar}\Matrix{\sqrt{1/2} & \sqrt{1/2}} \Matrix{0 \\ 1}\\
	&= \dfrac{4}{5\sqrt{2}} e^{-iE_{+}t/\hbar}- \dfrac{3}{5\sqrt{2}} e^{-iE_{-}t/\hbar}
\end{align*}
\subitem Similarly, $\Braket{-x \mid \psi(t)} = \dfrac{4}{5\sqrt{2}} e^{-iE_{+}t/\hbar} + \dfrac{3}{5\sqrt{2}} e^{-iE_{-}t/\hbar}$. Therefore, the probability of these out come is:
\begin{align*}
	|\Braket{+x \mid \psi(t)}|^{2} &= (\dfrac{4}{5\sqrt{2}} e^{-iE_{+}t/\hbar}- \dfrac{3}{5\sqrt{2}} e^{-iE_{-}t/\hbar})^{*}(\dfrac{4}{5\sqrt{2}} e^{-iE_{+}t/\hbar}- \dfrac{3}{5\sqrt{2}} e^{-iE_{-}t/\hbar})\\ 
	&=(\dfrac{4}{5\sqrt{2}} e^{iE_{+}t/\hbar} + \dfrac{3}{5\sqrt{2}} e^{iE_{-}t/\hbar})(\dfrac{4}{5\sqrt{2}} e^{-iE_{+}t/\hbar} + \dfrac{3}{5\sqrt{2}} e^{-iE_{-}t/\hbar})\\  
	&= \dfrac{1}{50}(16 + 12e^{i(E_{+} - E_{-})t / \hbar} + 12e^{i(E_{-} - E_{+})t / \hbar} + 9)\\
	&= \dfrac{1}{2} - \dfrac{12}{25} cos(\omega t)
\end{align*}
\begin{align*}
	|\Braket{-x \mid \psi(t)}|^{2} &= (\dfrac{4}{5\sqrt{2}} e^{-iE_{+}t/\hbar}+ \dfrac{3}{5\sqrt{2}} e^{-iE_{-}t/\hbar})^{*}(\dfrac{4}{5\sqrt{2}} e^{-iE_{+}t/\hbar} + \dfrac{3}{5\sqrt{2}} e^{-iE_{-}t/\hbar})\\
	&=\dfrac{1}{2} + \dfrac{12}{25} cos(\omega t)
\end{align*}
\subitem The expectation value for $S_{x}$ is:
\begin{align*}
	\Braket{S_{x}} &= \dfrac{\hbar}{2}(\dfrac{1}{2} - \dfrac{12}{25} cos(\omega t)) - \dfrac{\hbar}{2}(\dfrac{1}{2} + \dfrac{12}{25} cos(\omega t))\\
	&= -\dfrac{12}{25} \hbar cos(\omega t)
\end{align*}
\subitem For $\Braket{y}$ :
\begin{align*}
	\Braket{+y \mid \psi(t)} &= 4/5 e^{-iE_{+}t/\hbar}\Braket{+y \mid +z} -3/5 e^{-iE_{-}t/\hbar}\Braket{+y \mid -z}\\
	&= 4/5 e^{-iE_{+}t/\hbar}\Matrix{\sqrt{1/2} & -i\sqrt{1/2}} \Matrix{1 \\ 0} - 3/5 e^{-iE_{-}t/\hbar}\Matrix{\sqrt{1/2} & -i\sqrt{1/2}} \Matrix{0 \\ 1}\\
	&= \dfrac{4}{5\sqrt{2}} e^{-iE_{+}t/\hbar} + \dfrac{3}{5\sqrt{2}} i e^{-iE_{-}t/\hbar}
\end{align*}
\newpage
\subitem Similarly, $\Braket{-y \mid \psi(t)} = \dfrac{4}{5\sqrt{2}} e^{-iE_{+}t/\hbar} - \dfrac{3}{5\sqrt{2}} i e^{-iE_{-}t/\hbar}$. Therefore, the probability of these out come is:
\begin{align*}
	|\Braket{-y \mid \psi(t)}|^{2} &= (\dfrac{4}{5\sqrt{2}} e^{-iE_{+}t/\hbar} - \dfrac{3}{5\sqrt{2}} i e^{-iE_{-}t/\hbar})^{*}(\dfrac{4}{5\sqrt{2}} e^{-iE_{+}t/\hbar} - \dfrac{3}{5\sqrt{2}} i e^{-iE_{-}t/\hbar})\\ 
	&=(\dfrac{4}{5\sqrt{2}} e^{iE_{+}t/\hbar} + \dfrac{3}{5\sqrt{2}} i e^{-iE_{-}t/\hbar})(\dfrac{4}{5\sqrt{2}} e^{-iE_{+}t/\hbar} - \dfrac{3}{5\sqrt{2}} i e^{-iE_{-}t/\hbar})\\ 
	&= \dfrac{1}{2} + \dfrac{12}{25}i( e^{i(E_{+} - E_{-})t/\hbar} - e^{i(E_{-} - E_{+})t/\hbar})\\
	&= \dfrac{1}{2} + \dfrac{12}{25} sin(\omega t)
\end{align*}
\begin{align*}
	|\Braket{+y \mid \psi(t)}|^{2} &=(\dfrac{4}{5\sqrt{2}} e^{-iE_{+}t/\hbar} + \dfrac{3}{5\sqrt{2}} i e^{-iE_{-}t/\hbar})^{*}(\dfrac{4}{5\sqrt{2}} e^{-iE_{+}t/\hbar} + \dfrac{3}{5\sqrt{2}} i e^{-iE_{-}t/\hbar})\\
	&=\dfrac{1}{2} - \dfrac{12}{25} sin(\omega t)
\end{align*}
\subitem The expectation value for $S_{y}$ is:
\begin{align*}
	\Braket{S_{y}} &= \dfrac{\hbar}{2}(\dfrac{1}{2} - \dfrac{12}{25} sin(\omega t)) - \dfrac{\hbar}{2}(\dfrac{1}{2} - \dfrac{12}{25} sin(\omega t))\\
	&= -\dfrac{12}{25} \hbar sin(\omega t)
\end{align*}
\subitem for $\Braket{S_{z}}$:
\begin{align*}
	\Braket{+z \mid \psi(t)} &= 4/5 e^{-iE_{+}t/\hbar}\Braket{+z \mid +z} -3/5 e^{-iE_{-}t/\hbar}\Braket{+z \mid -z}\\
	&=4/5 e^{-iE_{+}t/\hbar}\\
	\Braket{-z \mid \psi(t)} &= 4/5 e^{-iE_{+}t/\hbar}\Braket{-z \mid +z} -3/5 e^{-iE_{-}t/\hbar}\Braket{-z \mid -z}\\
	&=-3/5 e^{-iE_{-}t/\hbar}
\end{align*}
\subitem Therefore, we can get the expectation value of $S_{z}$:
\begin{align*}
	\Braket{S_{z}} &= \dfrac{\hbar}{2} |\Braket{+z\mid \psi}|^2 - \dfrac{\hbar}{2} |\Braket{-z \mid \psi}|^2\\
	&=\dfrac{7}{50}\hbar
\end{align*}
\subitem According to the calculation, we know that the $\Braket{S_{z}}$ is a constant  and $\Braket{S_x}$ and $\Braket{S_y}$ are functions depend on time. We can easily find that S is rotating about z axis on the xy plane. This motion is what we expect in classic model. So this result make sense. 
\item QAD.5\\
\subitem According to the time evolution rule, we can assume that the energy eigenvector $\Ket{\psi_{n}} = c_{n}e^{-iE_{n}t/\hbar}\Ket{E_{n}}$, and we have a observable $\Ket{k}$.
\subitem so the probability is :
\begin{align*}
	&|\Braket{k\mid \psi_{n}}|^2 \\
	=&|\Braket{k\mid E_{n}}e^{-iE_{n}t/\hbar}|^2 \\
	=&(\Braket{k\mid E_{n}}^{*}c_{n}e^{iE_{n}t/\hbar})(\Braket{k\mid eE_{n}}c_{n}e^{-iE_{n}t/\hbar})\\
	=&c_{n}^{2}(\Braket{k\mid E_{n}}^{*})(\Braket{k\mid E_{n}})
\end{align*}
\subitem Because the probability does not depend on time, a energy eigenvector is a stationary state.
\end{enumerate}
\end{document}
 